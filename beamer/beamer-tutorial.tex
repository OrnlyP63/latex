\documentclass{beamer}
%\usepackage[none]{hyphenat}
\usepackage{multicol}

\usetheme[progressbar=frametitle]{metropolis}
\setbeamertemplate{frame numbering}[fraction]
\useoutertheme{metropolis}
\useinnertheme{metropolis}
\usefonttheme{metropolis}
\usecolortheme{spruce}
\setbeamercolor{background canvas}{bg=white}

\definecolor{mygreen}{rgb}{.125,.5,.25}
\usecolortheme[named=mygreen]{structure}

\title{Functions, Limits, Derivatives}
%\subtitle{Subtitle Here}
\author{}
\institute{\large \textbf{Learning Outcomes}: \\[6pt] Identify properties of elementary functions (formed by composition of power, exponential, logarithmic, and trigonometric functions and their inverses).}
\date{}

\setbeamercovered{transparent=5}

\begin{document}
\metroset{block=fill}

\begin{frame}
\titlepage
\end{frame}

\begin{frame}[t]{Functions} \vspace{4pt}
\begin{block}{Definition of a Function}
\vspace{0.5em}
A \textbf{function} $f$ is a rule that assigns to each element $x$ in a set $D$ exactly one element, called $f(x)$, in a set $E$.
\vspace{0.5em}
\end{block}

\vspace{10pt}
Set $D$ is called the 
\only<1>{ \line(1,0){50} }
\only<2>{\textcolor{magenta}{domain}}
 \, of the function.\\[10pt]

Set $E$ is called the 
\only<1>{ \line(1,0){50} }
\only<2>{\textcolor{magenta}{range}}
\, of the function.

\end{frame}

\end{document}
